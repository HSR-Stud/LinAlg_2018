%%%%%%%%%%%%%%%%%%%%%%%%%%%%%%%%%%%%%%%%%%%%%%%%
% Determinante
%%%%%%%%%%%%%%%%%%%%%%%%%%%%%%%%%%%%%%%%%%%%%%%%
\section{Determinante}

\subsection{Definition einer Determinante}
	\begin{enumerate}
		\item $det(A)$ ändert sich nicht unter der Operation $E$ bzw. \textit{blauen} Operation.
		\item Wird eine Zeile von $A$ mit $\lambda$ multipliziert, wird auch $det(A)$ mit $\lambda$ multipliziert: $det(\lambda A)=\lambda^{Anzahl Zeilen/Spalten}det(A)$
		\item $det(E) = 1$
	\end{enumerate}

\subsection{Eigenschaften der Determinante}
	\begin{itemize}
		\item Hat $A$ eine Nullzeile/Nullspalte, dann ist $det(A) = 0$
		\item Hat $A$ \underline{zwei gleiche} Zeilen/Spalten, dann ist $det(A) = 0$
		\item Ist $A$ regulär $\Leftrightarrow$ $det(A) \neq 0$ \\
			Ist $A$ singulär $\Leftrightarrow$ $det(A) = 0$
		\item Vertauscht man zwei Zeilen/Spalten, dann ändert sich das Vorzeichen der Determinante.
		\item Beschreibt eine Fläche eines Parallelogrammes (2D) bzw. ein Volumen eines Parallelepipeds (3D).
		\item Kann nur ermittelt werden, wenn die Matrix exkl. Lösungen quadratisch ist.
	\end{itemize}

\subsection{Gauss-Verfahren}
	 Um die Determinante mit dem Gauss-Verfahren zu bestimmen werden die rot umkreiste Pivot-Elemente herausgenommen und miteinander multipliziert.
	\begin{equation*}
		\left|\begin{array}{ccc}
			\circledr{2} & 2 & 4 \\
				4 & 5 & 6 \\
				7 & 8 & 5 \\
		\end{array}\right| 
		\rightarrow %^{Gauss} \rightarrow \; $$2{\;}* $$ {\;}
		2*
		\left|\begin{array}{ccc}	
			1 & 1 & 2 \\
			\circledb{4} & 5 & 6 \\
			\circledb{7} & 8 & 5 \\
		\end{array}\right|
		\rightarrow
		2*
		\left|\begin{array}{ccc}	
		1 & 1 & 2 \\
		0 & \circledr{1} & -2 \\
		0 & \circledb{1} & -9 \\
		\end{array}\right|
		\rightarrow
		2*1*
		\left|\begin{array}{ccc}	
		\circledr{1} & 1 & 2 \\
		0 & \circledr{1} & -2 \\
		0 & 0 & \circledr{-7} \\
		\end{array}\right|
		\rightarrow
		2*1*(-7)*1*1=-14
	\end{equation*}

\subsection{Entwicklungssatz}
	\begin{enumerate}
		\item Zeile/Spalte auswählen (mit möglichst vielen Nullen)
		\item 1 Element herausnehmen
		\item Zeile und Spalte des herausgenommenen Elements abdecken
		\item Element mit der Determinante der nicht abgedeckten Elemente multiplizieren
		\item Schritt 2-4 wiederholen und zum 1. Element addieren/subtrahieren ($\rightarrow$ siehe Vorzeichen Matrix)
	\end{enumerate}
	
	Vorzeichenmatrix: $\begin{array}{|c|c|c|c|}
		\hline + & - & + & - \\
		\hline - & + & - & + \\
		\hline + & - & + & - \\
		\hline - & + & - & + \\
		\hline \end{array}$ \\ \\

	\textbf{Beispiel:} $\left|\begin{array}{ccc}
		\color{red}a & \color{green}b & \color{blue}c \\
		d & e & f \\
		g & h & i \end{array}\right| 
	= {\color{red}a} \left|\begin{array}{cc}
		e & f \\
		h & i \end{array}\right| 
	- {\color{green}b} \left|\begin{array}{cc}
		d & f \\
		g & i \end{array}\right|
	+ {\color{blue}c} \left|\begin{array}{cc}
		d & e \\
		g & h \end{array}\right|$ \\

\subsection{Wichtige Determinanten}
	$det\left(\begin{array}{cc}
		a & b \\
		c & d \end{array}\right)
	= ad - bc \qquad \qquad
	det\left(\begin{array}{ccc}
		a & b & c \\
		d & e & f \\
		g & h & i \end{array}\right)
	= \underbrace{aei + bfg + cdh - ceg - afh - bdi}_{Sarrus'sche Formel}$

\subsection{Cramsche Regel}\label{Cramesche Regel}
	$x_1= \frac{\left|\begin{array}{cccc}
		b_1 & a_{12} & \ldots & a_{1n} \\
		\vdots & \vdots & \ddots & \vdots \\
		b_n & a_{n2} & \ldots & a_{nn} \end{array}\right|}{det(A)}; \qquad
	x_2 = \frac{\left|\begin{array}{ccccc}
		a_{11} & b_1 & a_{13} & \ldots & a_{1n}\\
		\vdots & \vdots & \vdots & \ddots & \vdots \\
		a_{n1} & b_n & a_{n3} & \ldots & a_{nn} \end{array}\right|}{det(A)}$ \\ \\
	Inverse Matrix mit Cramer (Minoren): $A^{-1} = C : c_{ik} = \frac{(-1)^{k+i} \cdot det(A_{ki})}{det(A)} \longrightarrow$ 1. Index = Zeile; 2.Index = Spalte
	
	\textbf{Beispiel 3x3 Matrix:} \ \ 
		$A=\left(\begin{array}{rrr} 
				a & b & c \\
				d & e & f \\
				g & h & i \\
			\end{array}\right)$ \\ \ \\
			
		$A^{-1}=\displaystyle \frac{1}{det(A)}
			\left(\begin{array}{rrr} 
				+\underbrace{\left|\begin{array}{rr} e & f \\ h & i \\ \end{array}\right|}_{det(A_{11})} &
				-\underbrace{\left|\begin{array}{rr} b & c \\ h & i \\ \end{array}\right|}_{det(A_{21})} &
				+\underbrace{\left|\begin{array}{rr} b & c \\ e & f \\ \end{array}\right|}_{det(A_{31})} \\
			
				-\underbrace{\left|\begin{array}{rr} d & f \\ g & i \\ \end{array}\right|}_{det(A_{12})} &
				+\underbrace{\left|\begin{array}{rr} a & c \\ g & i \\ \end{array}\right|}_{det(A_{22})} &
				-\underbrace{\left|\begin{array}{rr} a & c \\ d & f \\ \end{array}\right|}_{det(A_{32})} \\
			
				+\underbrace{\left|\begin{array}{rr} d & e \\ g & h \\ \end{array}\right|}_{det(A_{13})} &
				-\underbrace{\left|\begin{array}{rr} a & b \\ g & h \\ \end{array}\right|}_{det(A_{23})} &
				+\underbrace{\left|\begin{array}{rr} a & b \\ d & e \\ \end{array}\right|}_{det(A_{33})} \\
			\end{array}\right)$			
\subsection{Spezielle Fälle}
	$det\left(\begin{array}{cc} 
			A & 0 \\
			0 & B \\
		\end{array}\right)=det(A)det(B)$ \ \ wobei $A$ und $B$ Matrizen von Grösse $n*n$ sind.
\subsection{Produktsatz}
$det(C \cdot D) = det(C) \cdot det(D)$