%%%%%%%%%%%%%%%%%%%%%%%%%%%%%%%%%%%%%%%%%%%%%%%%
% Inverse
%%%%%%%%%%%%%%%%%%%%%%%%%%%%%%%%%%%%%%%%%%%%%%%%
\myexternaldocument{2_Determinante}
\clearpage
\section{Inverse}

\subsection{Definition einer Inverse}
	\begin{enumerate}
		\item Genau wie bei Determinanten gelten die Regeln hier auch.
	\end{enumerate}

\subsection{Eigenschaften der Inversen}
	\begin{itemize}
		\item Genau wie bei Determinanten besitzen Inversen die selben Eigenschaften, wobei einige Ausnahmen bestehen.
		\item Kann nur ermittelt werden, wenn das gilt: $det (A) \neq 0$.
	\end{itemize}

\subsection{Gauss-Verfahren}
	\begin{itemize}
		\item Mit dem Gauss-Verfahren kann das Ganze gelöst werden. Dazu schreibt man die Einheitsmatrix rechts neben der Matrix. Nach dem Gausse erhält man links die Einheitsmatrix und rechts die Inverse der Matrix.
			\begin{equation*}
				\begin{array}{|ccc|ccc|}
					\hline		
					2 & 2 & 4 & 1 & 0 & 0 \\
					4 & 5 & 6 & 0 & 1 & 0 \\
					7 & 8 & 5 & 0 & 0 & 1 \\
					\hline
					\multicolumn{3}{c}{\underbrace{\hphantom{st\kern4\tabcolsep 1}}_{A}} &
					\multicolumn{3}{c}{\underbrace{\hphantom{st\kern4\tabcolsep 1}}_{E}}
					\end{array}
				\rightarrow^{Gauss} \rightarrow
				\begin{array}{|ccc|ccc|}
				\hline		
					1 & 0 & 0 & {\displaystyle \frac{23}{14}} & {\displaystyle \frac{-11}{7}} & {\displaystyle \frac{4}{7}} \\
					0 & 1 & 0 & {\displaystyle \frac{-11}{7}} & {\displaystyle \frac{9}{7}} & {\displaystyle \frac{-2}{7}} \\
					0 & 0 & 1 & {\displaystyle \frac{3}{14}} & {\displaystyle \frac{1}{7}} & {\displaystyle \frac{-1}{7}} \\
				\hline 
				\multicolumn{3}{c}{\underbrace{\hphantom{st\kern4\tabcolsep 1}}_{E}} &
				\multicolumn{3}{c}{\underbrace{\hphantom{st\kern10\tabcolsep 1}}_{A^{-1}}}
				\end{array}
			\end{equation*}
	\end{itemize}

\subsection{Formeln für 2x2}
	\begin{equation*}
		A=\left(
		\begin{array}{ccc}
			a & b\\
			c & d\\
		\end{array}
		\right)
		\rightarrow
		A^{-1}=\displaystyle \frac{1}{det(A)}\left(
		\begin{array}{ccc}
		d & -b\\
		-c & a\\
		\end{array}
		\right)
		= \displaystyle \frac{1}{ad-bc}\left(
		\begin{array}{ccc}
		d & -b\\
		-c & a\\
		\end{array}
		\right)				
	\end{equation*} 

	