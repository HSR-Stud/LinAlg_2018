%%%%%%%%%%%%%%%%%%%%%%%%%%%%%%%%%%%%%%%%%%%%%%%%
% Lineare Gleichungssysteme
%%%%%%%%%%%%%%%%%%%%%%%%%%%%%%%%%%%%%%%%%%%%%%%%
\section{Lineare Gleichungssysteme}

%Für eine erklärung für den Gauss Algorithmus volgenden Befehl aktivieren
%\subsection{Gauss-Verfahren}
	Das Gauss-Verfahren ist ein Algorithmus und kann für das Lösen LGS verwendet werden. Es basiert auf das Additionsverfahren. Wobei die Schritte normiert sind. Damit hat man ein stures Verfahren, das bei sehr grossen Matrizen die Rechenschritte vereinfachen kann. \\ \\
	Vorgehen:
	\begin{itemize}
		\item Ein LGS mit 3 Gleichungen und 3 Unbekannten hat die Form: \\
			$$		a_{11}x_1 + a_{12}x_2 + a_{13}x_3 = b_1 $$
			$$		a_{21}x_1 + a_{22}y_2 + a_{23}x_3 = b_2 $$
			$$		a_{31}x_1 + a_{32}x_2 + a_{33}x_3 = b_3 $$
		\item Gleichungen aufstellen:
			$$ x + 4y + 3z = 8 $$
			$$ 2x + 8y + z = 7 $$
			$$ 3x + 2y + 5z = 5 $$
		\item Gleichungen in Gauss-Tableau überführen:
			\begin{equation*}
				\begin{bmatrix}[ccc|c]
					x_1		& x_2	 & x_3		& b_x \\
					\hline
					a_{11}	& a_{12} & a_{13}	& b_1 \\ 
					a_{21}	& a_{22} & a_{23}	& b_2 \\ 
					a_{31}	& a_{32} & a_{33}	& b_3
				\end{bmatrix}
				\rightarrow
				\begin{bmatrix}[ccc|c]
					1 & 4 & 3 & 8 \\ 
					2 & 8 & 1 & 7 \\ 
					3 & 2 & 5 & 5
				\end{bmatrix} 
				\rightarrow
				\begin{array}{|ccc|c|}
					\hline		
					1 & 2 & 3 & 8 \\
					4 & 5 & 6 & 7 \\
					7 & 8 & 5 & 5 \\
					\hline
				\end{array} 
			\end{equation*}
		\item Gauss Schritt für Schritt:
			\begin{itemize}
				\item Rot: {\quad}{\,} Die ganze Zeile mit rot umrandeter Zahl teilen.
				\item Blau:	{\quad}{\!} Die Zeile mit der vorher rot geteilten Zahl solange subtrahieren, bis 0 bei blau umrandeter Zahl steht.
				\item Orange: Einheitsmatrix bzw. jedes Element hat einen bestimmten Wert auf der rechten Seite bekommen.
			\end{itemize}
			\begin{equation*}
				\begin{array}{|ccc|c|}
					\hline		
					\circledr{2} & 2 & 4 & 8 \\
					4 & 5 & 6 & 7 \\
					7 & 8 & 5 & 5 \\
					\hline
				\end{array}
				\rightarrow^{Gauss} \rightarrow
				\begin{array}{|ccc|c|}
				\hline		
					1 & 1 & 2 & 4 \\
					\circledb{4} & 5 & 6 & 7 \\
					\circledb{7} & 8 & 5 & 5 \\
				\hline
				\end{array}
				\rightarrow^{Gauss} \rightarrow
				\begin{array}{|ccc|c|}
				\hline		
					1 & 1 & 2 & 4 \\
					0 & 1 & -2 & -9 \\
					0 & \circledb{1} & -9 & -23 \\
				\hline
				\end{array}
				\rightarrow^{Gauss} \rightarrow
				\begin{array}{|ccc|c|}
				\hline		
					1 & 1 & 2 & 4 \\
					0 & 1 & -2 & -9 \\
					0 & 0 & \circledr{-7} & -14 \\
				\hline
				\end{array}
				\rightarrow^{Gauss} \rightarrow
			\end{equation*}
			\begin{equation*}
				\begin{array}{|ccc|c|}
				\hline		
					1 & 1 & \circledb{2} & 4 \\
					0 & 1 & \circledb{-2} & -9 \\
					0 & 0 & 1 & 2 \\
				\hline
				\end{array}
				\rightarrow^{Gauss} \rightarrow
				\begin{array}{|ccc|c|}
				\hline		
					1 & \circledb{1} & 0 & 0 \\
					0 & 1 & 0 & -5 \\
					0 & 0 & 1 & 2 \\
				\hline
				\end{array}
				\rightarrow^{Gauss} \rightarrow
				\begin{array}{|ccc|c|}
				\hline		
					\circledy{1} & 0 & 0 & 5 \\
					0 & \circledy{1} & 0 & -5 \\
					0 & 0 & \circledy{1} & 2 \\
				\hline
				\end{array}
			\end{equation*}
	\end{itemize}

\subsection{mehere Gleichungssysteme simultan lösen}
	Wenn bei unterschiedlichen Lösungen immer das Gleiche auf der linken Seite steht, kann man mehrere LGS simultan lösen.
	
	Beim Lösen der Gleichungssysteme findet immer der gleiche Vorgang statt. 
	
	Daher können alle Lösungen bzw. Spalten gleich in die Gauss-Tableau mit integriert werden.\\
	
	Vorgehen:
	\begin{itemize}
		\item Für jede Gleichung die Lösungen bzw. Spalten auf der rechten Seite einfügen.
			\begin{itemize}
				\item Zum Beispiel 3x3 Matrix links und 3 Spalten.
			\end{itemize}
		\begin{equation*}
			\begin{bmatrix} 
				1 & 2 & 3 \\ 
				4 & 5 & 6 \\ 
				7 & 8 & 9 
			\end{bmatrix} 
			= 
			\begin{bmatrix} 
				a_{1} & a_{2} & a_{3} \\ 
				b_{1} & b_{2} & b_{3} \\ 
				c_{1} & c_{2} & c_{3}
			\end{bmatrix}
			\rightarrow^{Gauss} \rightarrow
			\begin{array}{|ccc|ccc|}
				\hline
				\color{red}x_1 & \color{red}x_2 & \color{red}x_3 & \color{green} b_1 & \color{green}b_2 & \color{green}b_3 \\
				\hline
					1 & 0 & 0 & b_{11} & b_{12} & b_{13}\\
					0 & 1 & 0 & b_{21} & b_{22} & b_{23}\\
					0 & 0 & 1 & b_{31} & b_{32} & b_{33}\\
				\hline
			\end{array}
		\end{equation*}
	\end{itemize}

\subsection{lineare Abhängigkeit}
	\begin{tabular}{ll}
		Koeffizienten: & $A = \left(\begin{array}{cc} 1 & 4\\ 2 & 8 \end{array}\right) \left\rbrace\begin{array}{l} l_1 = x +4y \\ l_2 = 2x + 8y \end{array}\right.$\\ \\
		Bestimmung $\lambda_i$:  &  $\lambda_1 l_1 = \lambda_1 (x + 4y) = \lambda_1 x + \lambda_1 4y$ \\
		& $\lambda_2 l_2 = \lambda_2 (2x + 8y) = \lambda_2 2x + \lambda_2 8y$
	\end{tabular} \\ \\
	
	\textbf{Def.:} Wenn $\lambda_1 l_1 + \lambda_2 l_2 = 0$, alle $\lambda_i = 0$ dann ist es linear \textbf{unabhängig} (= regulär). \\ \\

	$A^T = \left(\begin{array}{cc}
		1 & 2 \\
		4 & 8
	\end{array}\right) =  0 \Rightarrow \begin{array}{|cc|c|}
		\hline 1 & 2 & 0\\
		4 & 8 & 0\\
		\hline
	\end{array} \rightarrow^{Gauss} \rightarrow \begin{array}{|cc|c|}
		\hline 1 & 2 & 0\\
		0 & 0 & 0\\
		\hline
	\end{array} $ \qquad somit ist $\lambda_1 = -2\lambda_2 \rightarrow$ nicht alle $\lambda_i = 0 \Rightarrow$ lin. abhängig.\\ \\

	Falls A lin. unabhängig\\
	$ A^T = \left(\begin{array}{cc}
		3 & 2\\
		-6 & 4\\
	\end{array}\right) = 0 \Rightarrow \begin{array}{|cc|c|}
		\hline 3 & 2 & 0\\
		-6 & 4 & 0 \\
		\hline
	\end{array} \rightarrow^{Gauss} \rightarrow \begin{array}{|cc|c|}
		\hline 1 & 0 & 0\\
		0 & 1 & 0\\
		\hline
	\end{array}$  \qquad somit ist $\lambda_1 = 0, \lambda_2 = 0 \Rightarrow$ lin. unabhängig.\\

	Die Linieare Abhängigkeit kann geprüft werden, indem man bei einer Matrix den Gauss durchführt. Entsteht dabei eine \textbf{leer Zeile}	so ist es \textbf{linear abhängig} (= singulär). \\
	Sind \textbf{zwei gleiche} Zeilen- bzw. Spaltenvektoren in einer Matrix, so ist sie ebenfalls \textbf{linear abhängig}. \\
	Für eine linear abhänige Matrix gilt: \textbf{det(A)=0}


\subsection{Bezeichnung von Matrizen und Vektoren}
	\subsubsection{Vektoren}
		\begin{tabular}{ll}
			Zeilenvektor: & $v = \left(\begin{array}{cccc} a_1 & a_2 & \ldots & a_n \end{array}\right)$ \\
			Spaltenvektor: & $v = \left(\begin{array}{c} b_1 \\ b_2 \\ \vdots \\ b_m \end{array}\right)$ \\
			Nullvektor: & $v = \left(\begin{array}{cccc} 0 & 0 & \ldots & 0 \end{array}\right)$\\
			Einheitsvektor: & $e_1 = \left(\begin{array}{cccc} 1 & 0 & \ldots & 0 \end{array}\right) \qquad 
					e_2 = \left(\begin{array}{cccc} 0 & 1 & \ldots & 0 \end{array}\right)$
		\end{tabular}
	
	\subsubsection{Matrizen}
		\begin{tabular}{ll}
			Einheitsmatrix: & $E = \left(\begin{array}{cccc}
				1 & 0 & \ldots & 0 \\
				0 & 1 &  & \vdots \\
				\vdots &  & \ddots & 0\\
				0 & \ldots & 0 & 1 \end{array}\right)$ \\
			Inverse Matrix: & $A^{-1} \qquad \begin{array}{|c|c|} \hline A & E \\ \hline \end{array} \rightarrow^{Gauss} \rightarrow
					\begin{array}{|c|c|} \hline E & A^{-1} \\ \hline \end{array} $ \\
			Transponierte Matrix: & $A^T$ \qquad Zeilen und Spalten von A vertauschen	
		\end{tabular}

\subsection{Rang}
	Maximale Anzahl linear unabhängiger Zeilen (oder linear unabhängige Spalten).

\subsection{Homogen, Inhomogen}
	\begin{tabular}{ll}
		$Ax = b$ inhomogen & $Ax = 0$ homogen $\rightarrow b=0$\\
	\end{tabular}

	regulär $\left\lbrace\begin{array}{l}
		\text{homogen } \rightarrow \text{ Nulllösung } x=0\\
		\text{inhomogen } \rightarrow \text{ genau \underline{eine} Lösung }\end{array}\right.$ \\
	
	singulär $\left\lbrace\begin{array}{l}
		\text{homogen } Ax=0, b_1=0, b_2=0 \rightarrow \infty-\text{viele Lösungen}\\
		\text{inhomogen } Ax = b \left\lbrace\begin{array}{l}
			b_1 \neq 0, b_2 \neq 0 \rightarrow \text{ keine Lösung}\\
			b_1 \neq 0, b_2 =0 \rightarrow \infty-\text{viele Lösungen} \end{array}\right. \end{array}\right.$ \qquad 
		$ \begin{array}{|c | c|c|}
			\hline E & * & b_1 \\
			\hline 
			0 & 0 & b_2 \\
			\hline \end{array}$ \\

	Lösungsmenge eines inhomogenen Gleichungssystems mit $\infty$-vielen Lösungen\\
	$\rightarrow^{Gauss} \begin{array}{|ccc|c|}
		\hline 1 & 0 & -5 & 1 \\
		0 & 1 & 3 & 2\\
		0 & 0 & 0 & 0\\
		\hline \end{array}$ \begin{tabular}{l}
			$x = 1 + 5z$ \\
			$y = 2 - 3z$ \\
			$z = z$ \end{tabular} $\Rightarrow$ \begin{tabular}{l}
				$\mathbb{L}=\lbrace\left(\begin{array}{c}
					1+5z\\
					2-3z\\
					z \end{array}\right)\backslash z \in \mathbb{R} \rbrace$\\
				$\mathbb{L}=\lbrace\underbrace{\left(\begin{array}{c} 1 \\ 2 \\ 0 \end{array}\right)}_{x_p}
				+ \underbrace{z\left(\begin{array}{c} 5 \\ -3 \\ 1 \end{array}\right) \backslash z \in \mathbb{R}}_{\mathbb{L}_h} \rbrace$ \\
				$\mathbb{L}=\lbrace x_p + x_h \backslash x_h \in \mathbb{L}_h\rbrace$
			\end{tabular}\\

	$\mathbb{L}_h$  ist eine Gerade, Ebene... durch den Nullpunkt.


		