%Schriftgr"osse, Layout, Papierformat, Art des Dokumentes
\documentclass[10pt,twoside,a4paper,fleqn]{article}
%Einstellungen der Seitenr"ander
\usepackage[left=1cm,right=1cm,top=1cm,bottom=1cm,includeheadfoot]{geometry}
% Sprache, Zeichensatz, packages
\usepackage[utf8]{inputenc}
\usepackage[ngerman]{babel,varioref}
\usepackage{amssymb,amsmath,fancybox,graphicx,color,lastpage,wrapfig,fancyhdr,hyperref,verbatim,tabularx, textcomp}
%Text Umkreisung
\usepackage{tikz}
%Verweise auf anderes Kapitel
\usepackage{xr}
%\usepackage{hyperref} 

%pdf info
\hypersetup{pdfauthor={\authorinfo},pdftitle={\titleinfo},colorlinks=false}
%linkbordercolor=white
\author{\authorinfo}
\title{\titleinfo}

%Kopf- und Fusszeile
\pagestyle{fancy}
\fancyhf{}
%Linien oben und unten
\renewcommand{\headrulewidth}{0.5pt} 
\renewcommand{\footrulewidth}{0.5pt}

%Kopfzeile
\fancyhead[L]{\titleinfo{ }\tiny{(\versioninfo)}}
\fancyhead[R]{Seite \thepage { }von \pageref{LastPage}}

%Fusszeile
\fancyfoot[L]{\footnotesize{\authorinfo}}
\fancyfoot[C]{\footnotesize{\licence \quad $\rightarrow$ \href{https://github.com/HSR-Stud}{Github: HSR-Stud}}}
\fancyfoot[R]{\footnotesize{\today}}

%Neue Befehle:
%1: Fügt vertikaln Balken bei matrix ein.
%2: Umkreist den Text darin.
\makeatletter
\renewcommand*\env@matrix[1][*\c@MaxMatrixCols c]{%
	\hskip -\arraycolsep
	\let\@ifnextchar\new@ifnextchar
	\array{#1}}
\makeatother

\makeatletter
\newcommand*{\addFileDependency}[1]{% argument=file name and extension
	\typeout{(#1)}
	\@addtofilelist{#1}
	\IfFileExists{#1}{}{\typeout{No file #1.}}
}
\makeatother

\newcommand*{\myexternaldocument}[1]{%
	\externaldocument{#1}%
	\addFileDependency{#1.tex}%
	\addFileDependency{#1.aux}%
}

\newcommand*\circledr[1]	{
	\tikz [baseline = (char.base)]	{
		\node [shape = circle, draw = red, inner sep = 2pt, text = black] (char) {#1};
}
}

\newcommand*\circledb[1]	{
	\tikz [baseline = (char.base)]	{
		\node [shape = circle, draw = blue, inner sep = 2pt, text = black] (char) {#1};
}
}

\newcommand*\circledy[1]	{
	\tikz [baseline = (char.base)]	{
		\node [shape = circle, draw = orange, inner sep = 2pt, text = black] (char) {#1};
}
}